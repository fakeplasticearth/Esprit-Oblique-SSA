\documentclass[specialist, substylefile = spbureport.rtx, subf,href,colorlinks=true, 12pt]{disser}

\usepackage[a4paper,
            mag=1000, includefoot,
            left=3cm, right=1.5cm, top=2cm, bottom=2cm, headsep=1cm, footskip=1cm]{geometry}
\usepackage{mathtext}
\usepackage[T2A]{fontenc}
\usepackage[utf8]{inputenc}
\usepackage[english,russian]{babel}
\usepackage{amsmath}
\usepackage{amsthm}
\usepackage{bm}
\usepackage{float}
\usepackage{bbold}
\usepackage{dsfont}
\usepackage{graphicx}
\usepackage{ textcomp }
\usepackage{subcaption}
\usepackage{mathdots}
\usepackage{multirow}
\usepackage{longtable,booktabs,array}
\ifpdf\usepackage{epstopdf}\fi
\bibliographystyle{gost2008}

% Точка с запятой в качестве разделителя между номерами цитирований
%\setcitestyle{semicolon}

% Использовать полужирное начертание для векторов
\let\vec=\mathbf

% Включать подсекции в оглавление
\setcounter{tocdepth}{2}

\graphicspath{{fig/}}

\theoremstyle{definition}
\newtheorem{definition}{Определение}
\newtheorem{algorithm}{Алгоритм}
\newtheorem{remark}{Замечание}
\newtheorem{theorem}{Теорема}

\DeclareMathOperator{\rank}{rank}
\DeclareMathOperator{\diag}{diag}

\begin{document}
    Составим траекторную матрицу.
    \begin{equation*}
    \mathbf{S} = 
        \begin{pmatrix}
            c_1 & (c_1 + c_2)\mu & (c_1 + 2c_2)\mu^2 & \dots & (c_1 + (K - 1)c_2)\mu^{K - 1} \\
            (c_1 + c_2)\mu & (c_1 + 2c_2)\mu^2 & (c_1 + 3c_2)\mu^3 & \dots & (c_1 + Kc_2)\mu^K \\
            \vdots & \vdots & \vdots & \ddots & \dots \\
            (c_1 + (L - 1)c_2)\mu^{L - 1} & (c_1 + Lc_2)\mu^L & (c_1 + (L + 1)c_2)\mu^{L + 1} & \dots & (c_1 + (N - 1)c_2)\mu^{N - 1}
        \end{pmatrix}
    \end{equation*}
    Теперь нужно найти матрицу левых сингулярных векторов. Для этого составим ортонормальный базис пространства столбцов $\mathbf{S}$. Первым вектором возьмем \\ $f_1 := (1, \mu, \mu^2, \dots, \mu^{L - 1})^T$. Его необходимо нормировать. Посчитаем норму. 
\begin{equation*}
    c_1^2:=||(1, \mu, \mu^2, \dots, \mu^{L - 1})^T|| = \sum_{i = 0}^{L - 1}\mu^{2i}
\end{equation*}
В качестве второго вектора возьмем $e_2:=(0, \mu, 2\mu^2, 3\mu^3, \dots, (L - 1)\mu^{L - 1})^T$. Однако этот вектор не ортогонален первому. Построим ортогональный вектор $f_2$. Будем искать его как $f_2 := f_1 + \alpha e_2$. 
\begin{equation*}
    0 = (f_1, f_2) = (f_1, e_2) + \alpha (f_1, f_1) = (f_1, e_2) + \alpha c_1^2
\end{equation*}
\begin{equation*}
    (f_1, e_2) = \sum_{i = 0}^{L - 1}i \mu^{2i} \Rightarrow \alpha = -\frac{1}{c_1^2}\sum_{i = 0}^{L - 1}i\mu^{2i}
\end{equation*}
\begin{equation*}
    \Rightarrow f_2 = \begin{pmatrix}
        \alpha \\
        \mu (1 + \alpha) \\
        \mu^2(2 + \alpha) \\
        \mu^3(3 + \alpha) \\
        \dots \\
        \mu^{L - 1}(L - 1 + \alpha)
    \end{pmatrix}
\end{equation*}
Этот вектор необходимо еще нормировать. Для этого посчитаем норму. 
    \begin{equation*}
        n_2^2 = (f_2, f_2) = c_2^2 + 2\alpha(f_1, e_2) + \alpha^2c_1^2
    \end{equation*}
    Тем самым мы нашли матрицу $\mathbf{U}$, она состоит из двух векторов. Теперь найдем сдвиговую матрицу $\mathbf{M}$.
    \begin{equation*}
        \begin{pmatrix}
            \frac{1}{c_1} & \frac{\alpha}{n_2} \\
            \frac{\mu}{c_1} & \frac{\mu}{n_2}(1+ \alpha) \\
            \frac{\mu^2}{c_1} & \frac{\mu^2}{n_2}(2 + \alpha) \\
            \dots & \dots \\
            \frac{\mu^{L - 2}}{c_1} & \frac{\mu^{L - 2}}{n_2}(L - 2 + \alpha)
        \end{pmatrix}
        \begin{pmatrix}
            m_{11} & m_{12} \\
            m_{21} & m_{22}
        \end{pmatrix}
        =\begin{pmatrix}
            \frac{\mu}{c_1} & \frac{\mu}{n_2}(1 + \alpha) \\
            \frac{\mu^2}{c_1} & \frac{\mu ^ 2}{n_2}(2 + \alpha) \\
            \frac{\mu^3}{c_1} & \frac{\mu^3}{n_2}(3 + \alpha) \\
            \dots & \dots \\
            \frac{\mu^{L - 1}}{c_1} & \frac{\mu^{L - 1}}{n_2}(L - 1 + \alpha)
        \end{pmatrix}
    \end{equation*}
    Найдем сначала $m_{11}$ и $m_{21}$.
    \begin{equation*}
        \sum_{i = 0}^{L - 2}(\frac{\mu^i}{c_1}m_{11} + \frac{\mu^i(i + \alpha)}{n_2}m_{21} - \frac{\mu^{i + 1}}{c_1})^2 \longrightarrow extr
    \end{equation*}
    \begin{equation*}
        2\sum_{i = 0}^{L - 2}(\frac{\mu^i}{c_1}m_{11} + \frac{\mu^i(i + \alpha)}{n_2}m_{21} - \frac{\mu^{i + 1}}{c_1}) = 0 
    \end{equation*}
    Пусть $m_{11} = \mu$. Тогда получаем
    \begin{equation*}
        2m_{21}\sum_{i = 0}^{L - 2}\frac{\mu^i(i + \alpha)}{n_2} = 0 \Rightarrow m_{21} = 0.
    \end{equation*}
    Теперь найдем $m_{12}$ и $m_{22}$.
    \begin{equation*}
        \sum_{i = 0}^{L - 2}(\frac{\mu^i}{c_1}m_{12} + \frac{\mu^i(i + \alpha)}{n_2}m_{22} - \frac{\mu^{i + 1}(i + 1 + \alpha)}{n_2})^2 \longrightarrow extr
    \end{equation*}
    Пусть $m_{22} = \mu$. Тогда
    \begin{equation*}
         2m_{12}\sum_{i = 0}^{L - 2}\frac{\mu^i}{c_1} = 0 \Rightarrow m_{12} = \frac{c_1 \mu}{n_2}
    \end{equation*}
    Таким образом, получили матрицу $\mathbf{M}$.
    \begin{equation*}
        \mathbf{M} = 
        \begin{pmatrix}
            \mu & \frac{c_1 \mu}{n_2} \\
            0 & \mu
        \end{pmatrix}
    \end{equation*}
    Жорданов базис:
    \begin{equation*}
        \begin{pmatrix}
            \frac{c_1 \mu}{n_2} & 0 \\
            0 & 1
        \end{pmatrix}
    \end{equation*}
    В случае одной компоненты получаем разложение:
    \begin{equation*}
        (\mathbf{UT})(\mathbf{T}^{-1}\mathbf{\Sigma V}^T) = \mathbf{S}
    \end{equation*}
    Поэтому в случае одной компоненты, обобщенный EOSSA делает то же самое, что и обычный EOSSA. \\
    Матрица, полученная из спектрального разложения:
    \begin{equation*}
        \begin{pmatrix}
            -1 & 1 \\
            0 & 0
        \end{pmatrix}
    \end{equation*}
\end{document}